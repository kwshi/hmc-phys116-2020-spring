\documentclass{../phys116}

\title{Homework 8}
\author{}
\date{2020 April 6}

\begin{document}

\begin{exercise}[Townsend 7.3 (5 pts)]
  Show that properly normalized eigenstates of the harmonic oscillator
  are given by (7.37).  \textit{Suggestion:} Use induction.
\end{exercise}

\begin{solution}

\end{solution}

\begin{exercise}[Townsend 7.7 (5 pts)]
  A particle of mass \(m\) in the one-dimensional harmonic oscillator
  is in a state for which a measurement of the energy yields the
  values \(\frac{\hbar \omega}{2}\) or \(\frac{3 \hbar \omega}{2}\),
  each with a probability of one-half.  The average value of the
  momentum \(\abr{p_x}\) at time \(t=0\) is
  \(\sqrt{m \omega \hbar/2}\).  This information specifies the state
  of the particle completely.  What is this state and what is
  \(\abr{p_x}\) at time \(t\)?
\end{exercise}

\begin{solution}

\end{solution}

\begin{exercise}[Townsend 7.9 (5 pts)]
  Show that in the superposition of adjacent energy states (7.63) the
  average value of the position of the particle is given by
  \[
    \abr x = \bra \psi \hat x \ket \psi = A \cos (\omega t + \delta)
  \]
  and the average value of the momentum is given by
  \[
    \abr{p_x} = \bra \psi \hat p_x \ket \psi
    = - m \omega A \sin (\omega t + \delta)
  \]
  in accord with Ehrenfest's theorem, (6.33) and (6.34).
\end{exercise}

\begin{solution}

\end{solution}

\begin{exercise}[Townsend 7.14 (5 pts)]
  As shown in Section 7.1, for small oscillations a pendulum behaves
  as a simple harmonic oscillator.  Suppose a particle of mass \(m\)
  is in the ground state of a pendulum of length \(L\) and that
  instantaneously the length of the pendulum increases from \(L\) to
  \(4L\).  What is the probability the particle will be found to be in
  the ground state of this new oscillator?  Give a numerical answer.
\end{exercise}

\begin{solution}

\end{solution}

\end{document}