\documentclass {../phys116}

\title {Homework 10}
\author {}
\date {2020 April 20}

\begin {document}

\begin {exercise} [Townsend 9.9 (5 pts)]
  The carbon monoxide molecule, \ce {CO}, absorbs a photon with a
  frequency of \SI {1.15e11} {\Hz}, making a purely rotational
  transition from an \(l = 0\) to \(l = 1\) energy level.  What is the
  internuclear distance for this molecule?
\end {exercise}

\begin {solution}

\end {solution}

\begin {exercise} [Townsend 9.23 (5 pts)]
  The Hamiltonian for a three-dimensional system with \emph
  {cylindrical} symmetry is given by
  \[
    \hat H = \frac {\hat {\vec p}^2} {2 \mu} + V (\hat \rho)
  \]
  where \(\rho = \sqrt {x^2 + y^2}\).

  \begin {problems}
  \item Use symmetry arguments to establish that both \(\hat p_z\),
    the generate of translations in the \(z\) direction, and
    \(\hat L_z\), the generator of rotations about the \(z\) axis,
    commute with \(\hat H\).

  \item Use the fact that \(\hat H\), \(\hat p_z\), and \(\hat L_z\)
    have eigenstates in common to express the position-space
    eigenfunctions of the Hamiltonian in terms of those of
    \(\hat p_z\) and \(\hat L_z\).  \textit {Suggestion}: Follow a
    strategy similar to the one that we followed in (9.133) for a
    spherically symmetric potential except that here we are using the
    eigenfunctions of \(\hat p_z\) and \(\hat L_z\) instead of the
    eigenfunctions of \(\hat {\vec L}^2\) and \(\hat L_z\).

  \item What is the radial equation?  \textit {Note}: The Laplacian in
    cylindrical coordinates is given by
    \[
      \nabla^2 \psi
      = \frac 1 \rho \pdrv {} \rho \prn* {\rho \pdrv \psi \rho}
      + \frac 1 {\rho^2} \pdrv [2] \psi \phi + \pdrv [2] \psi z.
    \]
  \end {problems}
\end {exercise}

\begin {solution}

\end {solution}

\begin {exercise} [Townsend 9.12 (5 pts)]
  The wave function for a particle is of the form
  \(\psi (\vec r) = (x + y + z) f (r)\).  What are the values that a
  measurement of \(\vec L^2\) can yield?  What values can be obtained
  by measuring \(L_z\)?  What are the probabilities of obtaining these
  results?  \textit {Suggestion}: Express the wave function in
  spherical coordinates and then in terms of the \(Y_{l, m}\)s.
\end {exercise}

\begin {solution}

\end {solution}

\begin {exercise} [Townsend 9.20 (5 pts)]
  The wave function of a rigid rotator with a Hamiltonian
  \(\hat H = \frac {\hat {\vec L}^2} {2 I}\) is given by
  \[
    \braket {\theta, \phi} {\psi (0)}
    = \sqrt {\frac 3 {4 \pi}} \sin \theta \sin \phi.
  \]

  \begin {problems}
  \item What is \(\braket {\theta, \phi} {\psi(t)}\)?  \textit
    {Suggestion}: Express the wave function in terms of the
    \(Y_{l, m}\)s and the state in terms of \(\ket {l, m}\) kets.

  \item What values of \(L_z\) will be obtained if a measurement is
    carried out and with what probability will these values occur?

  \item What is \(\abr {L_x}\) for this state?  \textit {Suggestion}:
    Use bra-ket notation and express the operator \(\hat L_x\) in
    terms of raising and lowering operators.

  \item If a measurement of \(L_x\) is carried out, what result(s)
    will be obtained?  With what probability?  \textit {Suggestion}:
    If you have worked out Problem 3.15, you can take good advantage
    of the expressions for the states \(\ket {1, m}_x\).
    Specifically, take advantage of symmetry and
    \[
      \ket {1, 0}_x
      = \frac 1 {\sqrt 2} \ket {1, 1}
      - \frac 1 {\sqrt 2} \ket {1, -1}.
    \]
  \end {problems}
\end {exercise}

\begin {solution}

\end {solution}

\end {document}