\documentclass{../phys116}

\title{Homework 1}
\author{}
\date{2020 January 27}

\begin{document}

\begin{exercise}[Townsend 1.3, 5 pts]
  In Problem 3.2 we will see that the state of a spin-\(\frac 1 2\)
  particle that is spin up along the axis whose direction is specified
  by the unit vector in spherical coordinates
  \[
    \vec n
    = \sin \theta \cos \phi \, \vec i
    + \sin \theta \sin \phi \, \vec j
    + \cos \theta \, \vec k
  \]
  with \(\theta\) and \(\phi\) being the familiar spherical
  coordinates shown in Fig. 1.11, is given by
  \[
    \ket{+\vec n}
    = \cos \frac \theta 2 \, \ket{+\vec z}
    + e^{i\phi} \sin \frac \theta 2 \, \ket{-\vec z}.
  \]
  \begin{problems}
  \item Verify that the state \(\ket{+\vec n}\) reduces to the states
    \(\ket{+\vec x}\) and \(\ket{+\vec y}\) given in the chapter for
    the appropriate choice of the angles \(\theta\) and \(\phi\).
  \item Suppose that a measurement of \(S_z\) is carried out on a
    particle in the state \(\ket{+\vec n}\).  What is the probability
    that the measurement yields
    \begin{enumerate*}[label=(\roman*)]
    \item \(\hbar/2\)?
    \item \(-\hbar/2\)?
    \end{enumerate*}
  \end{problems}
\end{exercise}

\begin{solution}
\end{solution}

\begin{exercise}[Townsend 1.6+, 5 pts]
  Show that the state
  \[
    \ket{-\vec n}
    = \sin \frac \theta 2 \, \ket{+\vec z}
    - e^{i\phi} \cos \frac \theta 2 \, \ket{-\vec z}
  \]
  \begin{problems}
  \item really does correspond to \(\ket{+\vec n}\) with \(\vec n\)
    pointing in the opposite direction; \label{itm:opposite}
  \item is normalized such that \(\braket{-\vec n}{-\vec n} = 1\);
  \item is orthogonal to \(\ket{+\vec n}\).
  \item Show that as \(3\)-vectors, \((-\vec n) \cdot \vec n = -1\)
    \label{itm:negdot}.
  \end{problems}
  Note that \ref{itm:negdot} is consistent with having the quantum
  states be orthogonal: \(\braket{-n}{+n} = 0\).  The vector
  \(\vec n\) lives in physical 3D space whereas \(\ket{\vec n}\) lives
  in a 2D abstract ``Hilbert'' space.  The states \(\ket{\vec n}\) and
  \(\ket{-\vec n}\) are orthogonal because they give two different,
  definite, and mutually exclusive results when their spin is measured
  along the \(\vec n\) direction.

  \textit{Hint for part \ref{itm:opposite}: How shsould you change the
    polar angle to get a vector pointing in the opposite direction?}
\end{exercise}

\begin{solution}
\end{solution}

\end{document}