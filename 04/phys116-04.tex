\documentclass{../phys116}

\title{Homework 4}
\author{}
\date{2020 February 17}

\begin{document}

\begin{exercise}[Townsend 3.2 (5 pts)]
  Using the \(\ket{+\vec z}\) and \(\ket{-\vec z}\) states of a
  spin-\(\frac 1 2\) particle as a basis, set up and solve as a
  problem in matrix mechanics the eigenvalue problem for
  \(\hat S_n = \hat{\vec S} \cdot \vec n\), where the spin operator
  \(\hat{\vec S} = \hat S_x \vec i + \hat S_y \vec j + \hat S_k \vec
  k\) and
  \(\vec n = \sin \theta \cos \phi \, \vec i + \sin \theta \sin \phi
  \, \vec j + \cos \theta \, \vec k\).  Show that the eigenstates may
  be written as
  \begin{align*}
    \ket{+\vec n}
    &= \cos \frac \theta 2 \ket{+\vec z}
      + e^{i\phi} \sin \frac \theta 2 \ket{-\vec z}, \\
    \ket{-\vec n}
    &= \sin \frac \theta 2 \ket{+\vec z}
      - e^{i\phi} \cos \frac \theta 2 \ket{-\vec z}.
  \end{align*}
  Rather than simply verifying that these are eigenstates by
  substituting into the eigenvalue equation, obtain these states by
  directly solving the eigenvalue problem, as in Section 3.6.
\end{exercise}

\begin{solution}

\end{solution}

\begin{exercise}[Townsend 3.8 and 3.12 (5 pts)]
  Show that the operator \(\hat C\) defined through
  \(\sbr*{\hat A, \hat B} = i \hat C\) is Hermitian, provided that the
  operators \(\hat A\) and \(\hat B\) are Hermitian.  (This is fast.)

  Verify that for a spin-\(\frac 1 2\) particle that
  \begin{problems}
  \item
    \[
      \hat S_z
      = \frac \hbar 2 \ket{+\vec z} \bra{+\vec z}
      - \frac \hbar 2 \ket{-\vec z} \bra{-\vec z},
    \]
    and
  \item the raising and lowering operatos may be expressed as
    \[
      \hat S_+ = \hbar \ket{+\vec z} \bra{-\vec z} \qquad \text{and} \qquad
      \hat S_- = \hbar \ket{-\vec z} \bra{+\vec z}.
    \]
    \textit{Note}: It is sufficient to examine the action of these
    operators on the basis states \(\ket{+\vec z}\) and
    \(\ket{-\vec z}\), which of course form a complete set.
  \end{problems}
\end{exercise}

\begin{solution}

\end{solution}

\begin{exercise}[Townsend 3.15 and 3.16 (5 pts)]
  \begin{problems}
  \item Determine the eigenstates of \(\hat S_x\) for a spin-\(1\)
    particle in terms of the eigenstates \(\ket{1, 1}\),
    \(\ket{1, 0}\), and \(\ket{1, -1}\) of \(\hat S_z\).
  \item A spin-\(1\) particle exits an \(\SG_{\vec z}\) device in a
    state with \(S_z =\hbar\).  The beam then enters an
    \(\SG_{\vec x}\) device.  What is the probability that the
    measurement of \(S_x\) yields the value \(0\)?
  \end{problems}
\end{exercise}

\begin{solution}

\end{solution}

\end{document}