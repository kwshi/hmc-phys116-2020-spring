\documentclass{../phys116}

\title{Homework 7}
\author{}
\date{2020 March 9}

\begin{document}

\begin{exercise}[Townsend 5.21 (5 pts)]
  The density matrix for an ensemble of spin-\(\frac 1 2\) particles
  in the \(S_z\) basis is
  \[
    \hat \rho \xrightarrow[\text{\(S_z\) basis}]{}
    \begin{bmatrix} \frac 1 4  & n \\ n^* & p \end{bmatrix}.
  \]
  \begin{problems}
  \item What value must \(p\) have?  Why?
  \item What value(s) must \(n\) have for the density matrix to
    represent a pure state?
  \item What pure state is represented when \(n\) takes its maximum
    possible real value?  Express your answer in terms of the state
    \(\ket{+\vec n}\) given in Problem 5.19.
  \end{problems}
\end{exercise}

\begin{solution}
  \begin{problems}
  \item
  \item
  \item
  \end{problems}
\end{solution}

\begin{exercise}[Townsend 6.1 (5 pts)]
  \begin{problems}
  \item Use induction to show that
    \([\hat x^n, \hat p_x] = i \hbar n \hat x^{n-1}\).
    \textit{Suggestion}: Take advantage of the commutation relation
    \([\hat A \hat B, \hat C] = \hat A [\hat B, \hat C] + [\hat A,
    \hat C] \hat B\) in working out the commutators.
  \item Using the expansion
    \[
      F(x)
      = F(0)
      + \prn*{\drv F x}_{x=0} x
      + \frac{1}{2!} \prn*{\drv[2] F x}_{x=0} x^2 + \dots
      + \frac{1}{n!} \prn*{\drv[n] F x}_{x=0} x^n + \dots,
    \]
    show that
    \[
      [F(\hat x), \hat p_x] = i \hbar \, \pdrv F x (\hat x).
    \]
  \item For the one-dimensional Hamiltonian
    \[
      \hat H = \frac{\hat p_x^2}{2m} + V(\hat x)
    \]
    show that
    \[
      \drv{\abr{p_x}}{t} = \abr*{-\drv V x}.
    \]
  \end{problems}
\end{exercise}

\begin{solution}
  \begin{problems}
  \item
  \item
  \item
  \end{problems}
\end{solution}

\begin{exercise}[Townsend 6.2 (5 pts)]
  Show that
  \[
    \bra p \hat x \ket \psi = i \hbar \, \pdrv{}{p} \braket p \psi
  \]
  and
  \[
    \bra \varphi \hat x \ket \psi
    = \int \dif p \braket p \varphi^*
    i \hbar \, \pdrv{}{p} \braket p \psi.
  \]
  What do these results suggest for how you should represent the
  position operator in momentum space?
\end{exercise}

\begin{solution}
\end{solution}

\begin{exercise}[Townsend 6.4 (5 pts)]
  \begin{problems}
  \item Show for a free particle of mass \(m\) initially in the state
    \[
      \psi(x) = \braket x \psi
      = \frac{1}{\sqrt{\sqrt \pi a}} \, e^{-x^2/2a^2}
    \]
    that
    \[
      \psi(x, t) = \braket{x}{\psi(t)}
      = \frac{1}{\sqrt{\sqrt \pi (a + i \hbar t/m a)}}
      \, e^{-\frac{1}{2 (1 + i \hbar t/m a)} \frac{x^2}{a^2}}
    \]
    and therefore
    \[
      \Delta x = \frac{a}{\sqrt 2}
      \sqrt{1 + \prn*{\frac{\hbar t}{m a^2}}^2}.
    \]
    \textit{Suggestion}: Start with (6.75) and take advantage of the
    Gaussian integral (D.7), but in momentum space instead of position
    space.
  \item What is \(\Delta p\), at time \(t\)?  \textit{Suggestion}: Use
    the momentum-space wave function to evaluate \(\Delta p_x\).
  \end{problems}
\end{exercise}

\begin{solution}
  \begin{problems}
  \item
  \item
  \end{problems}
\end{solution}

\begin{exercise}[Townsend 6.5 (5 pts)]
  Consider a wave packet defined by
  \[
    \braket p \psi =
    \begin{cases}
      0 & p < -\frac P 2, \\
      N & -\frac P 2 < p < \frac P 2, \\
      0 & p > \frac P 2.
    \end{cases}
  \]
  \begin{problems}
  \item Determine a value for \(N\) such that
    \(\braket \psi \psi = 1\) using the momentum-space wave function
    directly.
  \item Determine \(\braket x \psi = \psi(x)\).
  \item Sketch \(\braket p \psi\) and \(\braket x \psi\).  Use
    reasonable estimates of \(\Delta p_x\) from the form of
    \(\braket p \psi\) and \(\Delta x\) from the form of
    \(\braket x \psi\) to estimate the product
    \(\Delta x \Delta p_x\).  Check that your result is independent of
    the value of \(P\).  \textit{Note}: Simply estimate rather than
    actually calculcate the uncertainties.
  \end{problems}
\end{exercise}

\begin{solution}
  \begin{problems}
  \item
  \item
  \item
  \end{problems}
\end{solution}

\begin{exercise}[Townsend 6.20 (5 pts)]
  Normalize the wave function
  \[
    \braket x \psi =
    \begin{cases}
      N e^{-\kappa x} & x > 0, \\
      N e^{\kappa x} & x < 0.
    \end{cases}
  \]
  Determine the probability that a measurement of the momentum \(p\)
  finds the momentum between \(p\) and \(p + \dif p\) for this wave
  function.  \textit{Note}: This wave function is the energy
  eigenfunction for the delta function potential energy well of
  Problem 6.19.
\end{exercise}

\begin{solution}
\end{solution}

\end{document}