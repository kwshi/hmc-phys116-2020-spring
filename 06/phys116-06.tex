\documentclass{../phys116}

\newcommand{\kv}[2]{\ket{#1 \vec{#2}}}
\newcommand{\pv}[2]{\proj{#1 \vec{#2}}}
\newcommand{\kvv}[3]{\ket{#2 \vec{#1}, #3 \vec{#1}}}
\newcommand{\bvv}[3]{\bra{#2 \vec{#1}, #3 \vec{#1}}}
\newcommand{\bkvv}[6]{\braket{#3 \vec{#1}, #4 \vec{#1}}{#5 \vec{#2}, #6 \vec{#3}}}
\newcommand{\xx}{\kvv x}
\newcommand{\nn}{\kvv n}
\newcommand{\zz}{\kvv z}

\title{Homework 6}
\author{}
\date{2020 March 2}

\begin{document}

\begin{exercise}[Townsend 5.2 (5 pts)]
  Express the total spin \(s=1\) states of two spin-\(\frac 1 2\)
  particles given in (5.30a) and (5.30c) in terms of the states
  \(\xx ++\), \(\xx +-\), and \(\xx --\).
\end{exercise}

\begin{solution}
\end{solution}

\begin{exercise}[Townsend 5.3 (5 pts)]
  Express the total-spin \(s=0\) state of two spin-\(\frac 1 2\)
  particles given in (5.31) in terms of the states \(\nn +-\) and
  \(\nn -+\), where for a single spin-\(\frac 1 2\) particle
  \begin{align*}
    \kv +n
    &= \cos \frac \theta 2 \, \ket{+\vec z}
    + e^{i\phi} \sin \frac \theta 2 \, \ket{-\vec z}, \\
    \kv -n
    &= \sin \frac \theta 2 \, \ket{+\vec z}
    - e^{i\phi} \cos \frac \theta 2 \, \ket{-\vec z}.
  \end{align*}
\end{exercise}

\begin{solution}
\end{solution}

\begin{exercise}[Townsend 5.5 (5 pts)]
  At time \(t=0\), an electron and a positron are formed in a state
  with total spin angular momentum equal to zero, perhaps from the
  decay of a spinless particle.  The particles are situated in a
  uniform magnetic field \(B_0\) in the \(z\) direction.
  \begin{problems}
  \item If interaction between the electron and positron may be
    neglected, show that the spin Hamiltonian of the system may be
    written as
    \[
      \hat H = \omega_0 (\hat S_{1z} - \hat S_{2z}),
    \]
    where \(\hat{\vec S}_1\) is the spin operator of the electron,
    \(\hat{\vec S}_2\) is the spin operator of the positron, and
    \(\omega_0\) is a constant.
  \item What is the spin state of the system at time \(t\)?  Show that
    the state of the system oscillates between a spin-\(0\) and
    spin-\(1\) state.  Determine the period of oscillation.
  \item At time \(t\), measurements are made of \(S_{1x}\) and
    \(S_{2x}\).  Calculate the probability that \emph{both} of these
    measurements yield the value \(\frac \hbar 2\).
  \end{problems}
\end{exercise}

\begin{solution}
  \begin{problems}
  \item
  \item
  \item
  \end{problems}
\end{solution}

\begin{exercise}[Townsend 5.6 (5 pts)]
  Take the spin Hamiltonian of the positronium atom (a bound state of
  an electron and a positron) in an external magnetic field in the
  \(z\) direction to be
  \[
    \hat H = \frac{2A}{\hbar^2} \, \hat{\vec S}_1 \cdot \hat{\vec S}_2
    + \omega_0 (\hat S_{1z} - \hat S_{2z}).
  \]
  Determine the energy eigenvalues.
\end{exercise}

\begin{solution}
\end{solution}

\end{document}
