\documentclass{../phys116}

\title{Homework 5}
\author{}
\date{2020 February 24}

\begin{document}

\begin{exercise}[Townsend 4.4 (5 pts)]
  A beam of spin-\(\frac 1 2\) particles with speed \(v_0\) passes
  through a series of two \(\SG_{\vec z}\) devices.  The first
  \(\SG_{\vec z}\) device transmits particles with
  \(S_z = \frac \hbar 2\) and filters out particles with
  \(S_z = -\frac \hbar 2\).  The second \(\SG_{mathrm z}\) device
  transmits particles with \(S_z = -\frac \hbar 2\) and filters out
  particles with \(S_z = \frac \hbar 2\).  Between the two devices is
  a region of length \(l_0\) in which there is a uniform magnetic
  field \(B_0\) pointing in the \(x\) direction.  Determine the
  smallest value of \(l_0\) such that exactly \(25\) percent of the
  particles transmitted by the first \(\SG_{\vec z}\) device are
  transmitted by the second device.  Express your result in terms of
  \(\omega_0 = \frac{egB_0}{2mc}\) and \(v_0\).
\end{exercise}

\begin{solution}

\end{solution}

\begin{exercise}[Townsend 4.5 (5 pts)]
  A beam of spin-\(\frac 1 2\) particles in the \(\ket{+\vec z}\)
  state enters a uniform magnetic field \(B_0\) in the \(x\)-\(z\)
  plane oriented at an angle \(\theta\) with respect to the \(z\)
  axis.  At time \(T\) later, the particles enter an \(\SG_{\vec y}\)
  device.  What is the probability the particles will be found with
  \(S_y = \frac \hbar 2\)?  Check your result by evaluating the
  special cases \(\theta = 0\) and \(\theta = \frac \pi 2\).
\end{exercise}

\begin{solution}

\end{solution}

\begin{exercise}[Townsend 4.8 (5 pts)]
  A spin-\(\frac 1 2\) particle, initially in a state with
  \(S_n = \frac \hbar 2\) with
  \(\vec n = \sin \theta \vec i + \cos \theta \vec k\), is in a
  constant magnetic field \(B_0\) in the \(z\) direction.  Determine
  the state of the particle at time \(t\) and determine how
  \(\abr{S_x}\), \(\abr{S_y}\), and \(\abr{S_z}\) vary with time.
\end{exercise}

\begin{solution}

\end{solution}

\begin{exercise}[Townsend 4.12 (5 pts)]
  A particle with intrinsic spin one is placed in a constant external
  magnetic field \(B_0\) in the \(x\) direction.  The initial spin
  state of the particle is \(ket{\phi(0)} = \ket{1, 1}\), that is, a
  state with \(S_z = \hbar\).  Take the spin Hamiltonian to be
  \[
    \hat H = \omega_0 \hat S_x,
  \]
  and determine the probability that the particle is in the state
  \(\ket{1, -1}\) at time \(t\).  \textit{Suggestion}: If you haven't
  already done so, you should first work out Problet 3.15 to determine
  the eigenstates of \(\hat S_x\) for a spin-\(1\) particle in terms
  of the eigenstates of \(\hat S_z\).
\end{exercise}

\begin{solution}

\end{solution}

\begin{exercise}[Townsend 4.13 (5 pts)]
  Let
  \[
    \begin{bmatrix}
      E_0 & 0   & A \\
      0   & E_1 & 0 \\
      A   & 0   & E_0
    \end{bmatrix}
  \]
  be the matrix representation of the Hamiltonian for a three-state
  system with basi states \(\ket 1\), \(\ket 2\), and \(\ket 3\).
  \begin{problems}
  \item If the state of the system at time \(t=0\) is
    \(\ket{\phi(0)} = \ket 2\), what is \(\ket{\phi(t)}\)?
  \item If the state of the system at time \(t=0\) is
    \(\ket{\phi(0)} = \ket 3\), what is \(\ket{\phi(t)}\)?
  \end{problems}
\end{exercise}

\begin{solution}
  \begin{problems}
  \item
  \item
  \end{problems}
\end{solution}

\end{document}