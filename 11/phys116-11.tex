\documentclass {../phys116}

\title {Homework 11}
\author {Kye W. Shi}
\date {2020 May 4}

\begin {document}

\begin {exercise} [Townsend 10.19 (5 pts)]
  The Hamiltonian for two spin-\(\frac 1 2\) particles, one with mass
  \(m_1\) and the other with mass \(m_2\), is given by
  \[
    \hat H
    = \frac {\hat {\vec p}_1^2} {2 m_1}
    + \frac {\hat {\vec p}_2^2} {2 m_2}
    + V_a (\abs {\hat {\vec r}})
    + \prn* {
      \frac 1 4
      - \frac {\hat {\vec S}_1 \cdot \hat {\vec S}_2} {\hbar^2}
    }
    V_b (\abs {\hat {\vec r}})
  \]
  where \(\hat {\vec r} = \hat {\vec r}_1 - \hat {\vec r}_2\) and
  \[
    V_a(r) =
    \begin {cases}
      0 & r < a, \\
      V_0 & r > a,
    \end {cases}
    \qquad
    V_b(r) =
    \begin {cases}
      0 & r < b, \\
      V_0 & r > b,
    \end {cases}
  \]
  with \(b < a\) and \(V_0\) very large and positive.
  \begin {problems}
  \item Determine the normalized position-space energy eigenfunction
    for the ground state.  What is the spin state of the ground state?
    What is the degeneracy?  \textit {Note}: Take \(V_0\) to be
    infinite.
  \item What can you say about the energy and spin state of the
    first-excited state?  Does your result depend on how much larger
    \(a\) is than \(b\)?  Explain.
  \end {problems}

  \textit {Suggestions}: Recall from Chapter 5 that the states of
  total spin are eigenstates of
  \(\hat {\vec S}_1 \cdot \hat {\vec S}_2\).  For spin-\(\frac 1 2\)
  particles we found that
  \begin {align*}
    \hat {\vec S}_1 \cdot \hat {\vec S}_2 \ket {s = 1, m} &= \frac {\hbar^2} 4, \\
    \hat {\vec S}_1 \cdot \hat {\vec S}_2 \ket {s = 0, 0} &= -\frac {3 \hbar^2} 4.
  \end {align*}
  Also, it may be useful to move to the center of mass frame and
  express the Hamiltonian in terms of \(\hat {\vec p}\) and
  \(\hat {\vec r}\), as we first did at the end of Lecture 17.
\end {exercise}

\begin {solution}
  \begin {problems}
  \item
  \end {problems}
\end {solution}

\begin {exercise} [Townsend 11.1 (5 pts)]
  Consider a perturbation \(\hat H_1 = b \hat x^4\) to the simple
  harmonic oscillator Hamiltonian
  \[
    \hat H_0
    = \frac {\hat p_x^2} {2 m}
    + \frac 1 2 m \omega^2 \hat x^2.
  \]
  This is an example of an anharmonic oscillator, one with a nonlinear
  restoring force.
  \begin {problems}
  \item Show that the first-order shift in the energy is given by
    \[
      E_n^{(1)} = \frac 3 4 \frac {\hbar^2 b} {m^2 \omega^2} (1 + 2 n + 2 n^2).
    \]
  \item Argue that no matter how small \(b\) is, the perturbation
    expansion will break down for some sufficiently large \(n\).  What
    is the physical reason?
  \end {problems}
\end {exercise}

\begin {solution}

\end {solution}

\begin {exercise} [Townsend 11.4 (5 pts)]
  Calculate the first-order shift to the energy of the ground state
  and first-excited state of a particle of mass \(m\) in the
  one-dimensional infinite square well
  \[
    V(x) =
    \begin {cases}
      0 & 0 < x < L, \\
      \infty & \text {elsewhere}.
    \end {cases}
  \]
  of
  \begin {problems}
  \item the constant perturbation \(\hat H_1 = V_1\) and
  \item the linearly increasing perturbation
    \(\hat H_1 = \epsilon E_1^{(0)} \hat x / L\), where \(E_1^{(0)}\)
    is the unperturbed energy of the ground state and
    \(\epsilon \ll 1\).
  \end {problems}
\end {exercise}

\begin {solution}
  \begin {problems}
  \item Since the perturbing Hamiltonian is constant, the first-order
    energy shifts are simply given by
    \[
      \bra {\psi_n^{(0)}} \hat H_1 \ket {\psi_n^{(0)}}
      = \bra {\psi_n^{(0)}} V_1 \ket {\psi_n^{(0)}}
      = V_1 \cancelto 1 {\braket {\psi_n^{(0)}} {\psi_n^{(0)}}}
      = V_1
    \]
    for any \(n\) (ground or first-excited, or whatever).

  \item Recall that the energy eigenstates of the
    infinite square well have wave functions given by
    \[
      \braket x {\psi_n^{(0)}}
      = \sqrt {\frac 2 L} \sin \frac {2 \pi x} {2 L / n}
      = \sqrt {\frac 2 L} \sin \frac {\pi n x} L
    \]
    for \(n = 1, 2, \dots\).

    Then we find the first-order energy shifts by taking inner
    products in position-space:
    \begin {align*}
      E_n^{(0)}
      &= \bra {\psi_n^{(0)}} \hat H_1 \ket {\psi_n^{(0)}} \\
      &= \frac 2 L \int_0^L \prn* {\sin \frac {\pi n x} L}^2
      \prn* {\epsilon E_1^{(0)} \frac x L} \dif x \\
      &= \frac {2 \epsilon E_1^{(0)}} {L^2} \int_0^L
      x \sin^2 \frac {\pi n x} L \dif x \\
      &= \frac {2 \epsilon E_1^{(0)}} {L^2} \int_0^L
      x \frac {1 - \cos \frac {2 \pi n x} L} 2 \dif x \\
      &= \frac {\epsilon E_1^{(0)}} {L^2}
      \int_0^L \prn* {x - x \cos \frac {2 \pi n x} L} \dif x \\
      &= \frac {\epsilon E_1^{(0)}} {L^2} \prn*{
        \prn* {\frac 1 2 x^2}_0^L
        - \frac L {2 \pi n}
        \int_0^L x \dif \sbr* {\sin \frac {2 \pi n x} L}
      } \\
      &= \frac {\epsilon E_1^{(0)}} {L^2} \prn* {
        \frac {L^2} 2
        - \frac L {2 \pi n} \prn* {
          \cancel {\sbr* {x \sin \frac {2 \pi n x} L}_0^L}
          - \cancel {\int_0^L \sin \frac {2 \pi n x} L \dif x}
        }
      } \\
      &= \frac \epsilon 2 E_1^{(0)}
    \end {align*}
    for arbitrary \(n\).
  \end {problems}
\end {solution}

\begin {exercise} [Townsend 11.15 (5 pts)]
  Determine the effect of an external magnetic field on the energy
  levels of the \(n=2\) states of hydrogen when the applied magnetic
  field \(B\) has a magnitude much greater than \SI {e4} {gauss}, in
  which case the spin-orbit interaction may be neglected as a first
  approximation.  This is the \textbf {Pashen-Bach effect}.

  Draw an energy-splitting diagram like Figure 11.10.

  \textit {Hint}: Unlike most of the section on the Zeeman Effect, you
  can start with the (11.105) form rather than the (11.106) form of
  the perturbing Hamiltonian.
\end {exercise}

\begin {solution}
  We have
  \[
    \hat H_B = \frac {eB} {2 m_e c} (\hat L_z + 2 \hat S_z).
  \]

\end {solution}

\begin {exercise} [Townsend 12.2 (5 pts)]
  Two identical, noninteracting spin-\(\frac 1 2\) particles of mass
  \(m\) are in the one-dimensional harmonic oscillator for which the
  Hamiltonian is
  \[
    \hat H
    = \frac {\hat p_{1x}^2} {2m} + \frac 1 2 m \omega^2 \hat x_1^2
    + \frac {\hat p_{2x}^2} {2m} + \frac 1 2 m \omega^2 \hat x_2^2.
  \]
  \begin {problems}
  \item Determine the ground-state and first-excited-state kets and
    corresponding energies when the two particles are in a
    total-spin-\(0\) state.  What are the lowest energy states and
    corresponding kets for the particles if they are in a
    total-spin-\(1\) state?
  \item Suppose the two particles interact with a potential energy of
    interaction
    \[
      V(\abs {x_1 - x_2}) =
      \begin {cases}
        -V_0 & \abs {x_1-x_2} < a, \\
        0 & \text {elsewhere}.
      \end {cases}
    \]
    Argue what the effect will be on the energies that you determined
    in (a), that is, whether the energy of each state moves up, moves
    down, or remains unchanged.  \textit {Suggestion}: Examine which
    spatial wave functions for the total-spin-\(0\) and
    total-spin-\(1\) states tend to have the particles closer
    together.  Consider, for example, the special case of
    \(x_1 = x_2\).
  \end {problems}
\end {exercise}

\begin {solution}

\end {solution}

\begin {exercise} [Townsend 12.4 (5 pts)]
  Use the variational principle to estimate the ground-state energy
  for the one-dimensional anharmonic oscillator
  \[
    \hat H = \frac {\hat p_x^2} {2m} + b \hat x^4.
  \]
  Compare your result with the exact result
  \[
    E_0 = 1.060 b^{1/3} \prn* {\frac {\hbar^2} {2m}}^{2/3}.
  \]
\end {exercise}

\begin {solution}

\end {solution}

\end {document}