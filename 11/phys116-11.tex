\documentclass {../phys116}

\title {Homework 11}
\author {Kye W. Shi}
\date {2020 May 4}

\begin {document}

\begin {exercise} [Townsend 10.19 (5 pts)]
  The Hamiltonian for two spin-\(\frac 1 2\) particles, one with mass
  \(m_1\) and the other with mass \(m_2\), is given by
  \[
    \hat H
    = \frac {\hat {\vec p}_1^2} {2 m_1}
    + \frac {\hat {\vec p}_2^2} {2 m_2}
    + V_a (\abs {\hat {\vec r}})
    + \prn* {
      \frac 1 4
      - \frac {\hat {\vec S}_1 \cdot \hat {\vec S}_2} {\hbar^2}
    }
    V_b (\abs {\hat {\vec r}})
  \]
  where \(\hat {\vec r} = \hat {\vec r}_1 - \hat {\vec r}_2\) and
  \[
    V_a(r) =
    \begin {cases}
      0 & r < a, \\
      V_0 & r > a,
    \end {cases}
    \qquad
    V_b(r) =
    \begin {cases}
      0 & r < b, \\
      V_0 & r > b,
    \end {cases}
  \]
  with \(b < a\) and \(V_0\) very large and positive.
  \begin {problems}
  \item Determine the normalized position-space energy eigenfunction
    for the ground state.  What is the spin state of the ground state?
    What is the degeneracy?  \textit {Note}: Take \(V_0\) to be
    infinite.
  \item What can you say about the energy and spin state of the
    first-excited state?  Does your result depend on how much larger
    \(a\) is than \(b\)?  Explain.
  \end {problems}

  \textit {Suggestions}: Recall from Chapter 5 that the states of
  total spin are eigenstates of
  \(\hat {\vec S}_1 \cdot \hat {\vec S}_2\).  For spin-\(\frac 1 2\)
  particles we found that
  \begin {align*}
    \hat {\vec S}_1 \cdot \hat {\vec S}_2 \ket {s = 1, m} &= \frac {\hbar^2} 4, \\
    \hat {\vec S}_1 \cdot \hat {\vec S}_2 \ket {s = 0, 0} &= -\frac {3 \hbar^2} 4.
  \end {align*}
  Also, it may be useful to move to the center of mass frame and
  express the Hamiltonian in terms of \(\hat {\vec p}\) and
  \(\hat {\vec r}\), as we first did at the end of Lecture 17.
\end {exercise}

\begin {solution}
  \begin {problems}
  \item
  \end {problems}
\end {solution}

\begin {exercise} [Townsend 11.1 (5 pts)]
  Consider a perturbation \(\hat H_1 = b \hat x^4\) to the simple
  harmonic oscillator Hamiltonian
  \[
    \hat H_0
    = \frac {\hat p_x^2} {2 m}
    + \frac 1 2 m \omega^2 \hat x^2.
  \]
  This is an example of an anharmonic oscillator, one with a nonlinear
  restoring force.
  \begin {problems}
  \item Show that the first-order shift in the energy is given by
    \[
      E_n^{(1)} = \frac 3 4 \frac {\hbar^2 b} {m^2 \omega^2} (1 + 2 n + 2 n^2).
    \]
  \item Argue that no matter how small \(b\) is, the perturbation
    expansion will break down for some sufficiently large \(n\).  What
    is the physical reason?
  \end {problems}
\end {exercise}

\begin {solution}

\end {solution}

\begin {exercise} [Townsend 11.4 (5 pts)]
  Calculate the first-order shift to the energy of the ground state
  and first-excited state of a particle of mass \(m\) in the
  one-dimensional infinite square well
  \[
    V(x) =
    \begin {cases}
      0 & 0 < x < L, \\
      \infty & \text {elsewhere}.
    \end {cases}
  \]
  of
  \begin {problems}
  \item the constant perturbation \(\hat H_1 = V_1\) and
  \item the linearly increasing perturbation
    \(\hat H_1 = \epsilon E_1^{(0)} \hat x / L\), where \(E_1^{(0)}\)
    is the unperturbed energy of the ground state and
    \(\epsilon \ll 1\).
  \end {problems}
\end {exercise}

\begin {solution}

\end {solution}

\begin {exercise} [Townsend 11.15 (5 pts)]
  Determine the effect of an external magnetic field on the energy
  levels of the \(n=2\) state of hydrogen when the applied magnetic
  field \(B\) has a magnitude much greater than \SI {e4} {gauss}, in
  which case the spin-orbit interaction may be neglected as a first
  approximation.  This is the \textbf {Pashen-Bach effect}.

  Draw an energy-splitting diagram like Figure 11.10.

  \textit {Hint}: Unlike most of the section on the Zeeman Effect, you
  can start with the (11.105) form rather than the (11.106) form of
  the perturbing Hamiltonian.
\end {exercise}

\begin {solution}

\end {solution}

\end {document}