\documentclass{../phys116}

\title{Homework 3}
\author{}
\date{2020 February 10}

\begin{document}

\begin{exercise}[Townsend 2.5 (5 pts)]
  What is the matrix representation of \(\hat J_z\) using the states
  \(\ket{+\vec y}\) and \(\ket{-\vec y}\) as a basis?  Use this
  representation to evaluate the expectation value of \(S_z\) for a
  collection of particles each in the state \(\ket{-\vec y}\).
\end{exercise}

\begin{solution}
\end{solution}

\begin{exercise}[Townsend 2.12 (5 pts)]
  A photon polarization state for a photon propagating in the \(z\)
  direction is given by
  \[
    \ket \psi
    = \sqrt{\frac 2 3} \ket{\mathrm x}
    + \frac i {\sqrt 3} \ket{\mathrm y}
  \]
  \begin{problems}
  \item What is the probability that a photon in this state will pass
    through an ideal polarizer with its transmission axis oriented in
    the \(y\) direction?

  \item What is the probability that a photon in this state will pass
    through an ideal polarizer with its transmission axis \(y'\)
    making an angle \(\phi\) with the \(y\) axis?

  \item A beam carrying \(N\) photons per second, each in the state
    \(\ket \psi\), is totally absorbed by a black disk with its normal
    to the surface in the \(z\) direction.  How large is the torque
    exerted on the disk?  In which direction does the disk rotate?
    \textit{Reminder}: The photon states \(\ket{\mathrm R}\) and
    \(\ket{\mathrm L}\) each carry a unit \(\hbar\) of angular
    momentum parallel and antiparallel, respectively, to the direction
    of propagation of the photons.

  \item How would the result for each of these questions differ if the
    polarization state were
    \[
      \ket{\psi'}
      = \sqrt{\frac 2 3} \ket{\mathrm x}
      + \frac{1}{\sqrt 3} \ket{\mathrm y},
    \]
    that is, the ``\(i\)'' in the state \(\ket \psi\) is absent?
  \end{problems}
\end{exercise}

\begin{solution}
  \begin{problems}
  \item
  \item
  \item
  \item
  \end{problems}
\end{solution}

\begin{exercise}[Townsend 2.13 (5 pts)]
  A system of \(N\) ideal linear polarizers is arranged in sequence,
  as shown in Fig. 2.13.  The transmission axis of the first polarizer
  makes an angle of \(\frac \phi N\) with the \(y\) axis.  The
  transmission axis of every other polarizer makes an angle of
  \(\frac \phi N\) with respect to the axis of the precedinig one.
  Thus, the transmission axis of the final polarizer makes an angle
  \(\phi\) with the \(y\) axis.  A beam of \(y\)-polarized photons is
  incident on the first polarizer.
  \begin{problems}
  \item What is the probability that an incident photon is transmitted
    by the array?
  \item Evaluate the probability of transmission in the limit of large
    \(N\).
  \item Consider the special case with the angle
    \(\phi = \SI{90}{\degree}\).  Explain why your result is not in
    conflict with the fact that \(\braket{\mathrm x}{\mathrm y} = 0\).
  \end{problems}
\end{exercise}

\begin{solution}
  \begin{problems}
  \item
  \item
  \item
  \end{problems}
\end{solution}

\begin{exercise}[Townsend 2.21 (5 pts)]
  Linearly polarized light of wavelength \SI{5890}{\angstrom} is
  incident normally on a birefringent crystal that has its optic axis
  parallel to the face of othe crystal, along the \(x\) axis.  If the
  incident light is polarized at an angle of \SI{45}{\degree} to the
  \(x\) and \(y\) axes, what is the probability that the photons
  exiting a crystal of thickness \num{100.0} microns will be
  right-circularly polarized?  The index of refraction for light of
  this wavelength polarized along \(y\) (perpendicular to the optic
  axis) is \num{1.66} and the index of refraction for light polarized
  along \(x\) (parallel to the optic axis) is \num{1.49}.
\end{exercise}

\begin{solution}
\end{solution}

\end{document}