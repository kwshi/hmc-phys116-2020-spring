\documentclass {../phys116}

\title {Homework 9}
\author {}
\date {2020 April 13}

\begin {document}

\begin {exercise} [Townsend 9.1 (5 pts)]
  Follow the suggestion after (9.11) to show that
  \([\hat T (a_x \vec i), \hat T (a_x \vec j)] = 0\) implies
  \([\hat p_x, \hat p_y] = 0\).
\end {exercise}

\begin {solution}

\end {solution}

\begin {exercise} [Townsend 9.5 (5 pts)]
  Show explicitly that
  \[
    \frac {\hat {\vec p}_1^2} {2 m_1}
    + \frac {\hat {\vec p}_2^2} {2 m_2}
    = \frac {\hat {\vec P}^2} {2 M}
    + \frac {\hat {\vec p}^2} {2 \mu},
  \]
  where
  \[
    \hat {\vec p}
    = \frac {m_2 \hat {\vec p}_1 - m_1 \hat {\vec p}_2} {m_1 + m_2},
    \qquad \hat {\vec P} = \hat {\vec p}_1 + \hat {\vec p}_2,
    \qquad M = m_1 + m_2,
    \qquad \text {and} \qquad \mu = \frac {m_1 m_2} {m_1 + m_2}.
  \]
  Then show that the total orbital angular momentum of the two
  particles about the center of mass position is the same as the
  orbital angular momentum about the center of mass position of a
  single particle that has position \(\vec r = \vec r_1 - \vec r_2\)
  and momentum \(\vec p\).
\end {exercise}

\begin {solution}

\end {solution}

\begin {exercise} [Townsend 9.6 (5 pts)]
  Use the fact that all 3D vectors must transform the same way under
  3D rotations to establish that any vector operator \(\hat {\vec V}\)
  must satisfy the below commutation relations:
  \[
    [\hat L_z, \hat V_x] = i \hbar \hat V_y,
    \qquad [\hat L_z, \hat V_y] = -i \hbar \hat V_x,
    \qquad [\hat L_z, \hat V_z] = 0.
  \]
  \textit {Suggestion:} Consider the commutations of either the
  position or momentum operators.
\end {exercise}

\begin {solution}

\end {solution}

\begin {exercise} [Townsend 9.13 (5 pts)]
  A particle is in the orbital angular momentum state \(\ket {l, m}\).
  Evaluate \(\Delta L_x\) and \(\Delta L_y\) for this state.  Which
  states satisfy the equality in the uncertainty relation
  \[
    \Delta L_x \Delta L_y \ge \frac \hbar 2 \abs {\abr {L_z}}?
  \]
  \textit {Suggestion:} One approach is to use
  \(\hat L_x = (\hat L_+ + \hat L_-)/2\), and so on.  Another is to
  take advantage of the symmetry of the expectation values of
  \(L_x^2\) and \(L_y^2\) in an eigenstate of \(\hat L_z\).

  \textit {Note:} This could have been a question about general
  angular momentum---nothing is specific here to orbital.
\end {exercise}

\begin {solution}

\end {solution}

\end {document}