\documentclass{../phys116}

\title{Homework 2}
\author{}
\date{2020 January 31}

\begin{document}

\begin{exercise}[Townsend 1.5 (5 pts)]
  \begin{problems}
  \item What is the amplitude to find a particle that is in the state
    \(\ket{+\vec n}\) (from Problem 1.3) with \(S_y = \frac \hbar 2\)?
    What is the probability?  Check your result by evaluating the
    probability for an appropriate choice of the angles \(\theta\) and
    \(\phi\).
  \item What is the amplitude to find a particle that is in the state
    \(\ket{+\vec y}\) with \(S_n = \frac \hbar 2\)?  What is the
    probability?
  \end{problems}
\end{exercise}

\begin{solution}
\end{solution}

\begin{exercise}[Townsend 1.7 (5 pts)]
  \newcommand{\SG}{\mathrm{SG}}

  A beam of spin-\(\frac 1 2\) particles is sent through a series of
  three Stern--Gerlach measuring devices, as illustrated in Fig. 1.12.
  The first \(\SG_{\vec z}\) devices transmits particles with
  \(S_z = \frac \hbar 2\) and filters out particles with
  \(S_z = -\frac \hbar 2\).  The second device, an \(\SG_{\vec n}\)
  device, transmits particles with \(S_n = \frac \hbar 2\) and filters
  out particles with \(S_n = -\frac \hbar 2\), where the axis
  \(\vec n\) marks an angle \(\theta\) in the \(x\)-\(z\) plane with
  respect to the \(z\) axis.  Thus particles after passage through
  this \(\SG_{\vec n}\) device are in the state \(\ket{+\vec n}\)
  given in Problem 1.3 with the angle \(\phi = 0\).  A last
  \(\SG_{\vec z}\) device transmits particles with
  \(S_z = -\frac \hbar 2\) and filters out particles with
  \(S_z = \frac \hbar 2\).
  \begin{problems}
  \item What fraction of the particles transmitted by the first
    \(\SG_{\vec z}\) device will survive the third measurement?
  \item How must the angle \(\theta\) of the \(\SG_{\vec n}\) device
    be oriented so as to maximize the number of particles that are
    transmitted by the final \(\SG_{\vec z}\) device?  What fraction
    of the particles survive the third measurement for this value of
    \(\theta\)?
  \item What fraction of the particles survive the last measurement if
    the \(\SG_{\vec n}\) device is simply removed from the experiment?
  \end{problems}
\end{exercise}

\begin{solution}
\end{solution}

\begin{exercise}[Townsend 1.15 (5 pts)]
  It is known that there is a \SI{90}{\percent} probability of
  obtaining \(S_z = \frac \hbar 2\) if a measurement of \(S_z\) is
  carried out on a spin-\(\frac 1 2\) particle.  In addition, it is
  known that there is a \SI{20}{\percent} probability of obtaining
  \(S_y = \frac \hbar 2\) if a measurement of \(S_y\) is carried out.
  Determine the spin state of the particle as completely as possible
  from this information.  What is the probability of obtaining
  \(S_x = \frac \hbar 2\) if a measurement of \(S_x\) is carried out?
\end{exercise}

\begin{solution}
\end{solution}

\begin{exercise}[Townsend 2.8 (5 pts)]
  The column vector representing the state \(\ket \psi\) is given by
  \[
    \ket \psi \xrightarrow[\text{\(S_z\) basis}]{} \frac{1}{\sqrt 5}
    \begin{pmatrix} i \\ 2 \end{pmatrix}.
  \]
  Using matrix mechanics, show that \(\ket \psi\) is properly
  normalized and calculate the probability that a measurement of
  \(S_x\) yields \(\frac \hbar 2\).  Also determine the probability
  that a measurement of \(S_y\) yields \(\frac \hbar 2\).
\end{exercise}

\begin{solution}
\end{solution}

\begin{exercise}[Townsend 2.6 (5 pts)]
  Evaluate \(\hat R(\theta \vec j) \ket{+\vec z}\), where
  \(\hat R(\theta \vec j) = e^{-i \hat J_y \theta/\hbar}\) is the
  operator that rotates kets counterclockwise by angle \(\theta\)
  about the \(y\) axis.  Show that
  \(\hat R\prn*{\frac \pi 2 \vec j} \ket{+\vec z} = \ket{+\vec x}\).
  \textit{Suggestion:} Express the ket \(\ket{+\vec z}\) as a
  superposition of the kets \(\ket{+\vec y}\) and \(\ket{-\vec y}\)
  and take advantage of the fact that
  \(\hat J_y \ket{\pm \vec y} = \prn*{\pm \frac \hbar 2} \ket{\pm \vec
    y}\); then switch back to the \(\ket{+\vec z}\) and
  \(\ket{-\vec z}\) basis.
\end{exercise}

\begin{solution}
\end{solution}

\end{document}